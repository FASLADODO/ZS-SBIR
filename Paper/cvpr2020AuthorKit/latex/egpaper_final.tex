\documentclass[10pt,twocolumn,letterpaper]{article}

\usepackage{cvpr}
\usepackage{times}
\usepackage{epsfig}
\usepackage{graphicx}
\usepackage{amsmath}
\usepackage{amssymb}

% Include other packages here, before hyperref.

% If you comment hyperref and then uncomment it, you should delete
% egpaper.aux before re-running latex.  (Or just hit 'q' on the first latex
% run, let it finish, and you should be clear).
\usepackage[breaklinks=true,bookmarks=false]{hyperref}

\cvprfinalcopy % *** Uncomment this line for the final submission

\def\cvprPaperID{****} % *** Enter the CVPR Paper ID here
\def\httilde{\mbox{\tt\raisebox{-.5ex}{\symbol{126}}}}

% Pages are numbered in submission mode, and unnumbered in camera-ready
%\ifcvprfinal\pagestyle{empty}\fi
\setcounter{page}{4321}
\begin{document}

%%%%%%%%% TITLE
\title{Zero-Shot Sketch-Based Image Retrieval with Disentangled Representation}

%\author{First Author\\
%Institution1\\
%Institution1 address\\
%{\tt\small firstauthor@i1.org}
%\and
%Second Author\\
%Institution2\\
%First line of institution2 address\\
%{\tt\small secondauthor@i2.org}
%}
% For a paper whose authors are all at the same institution,
% omit the following lines up until the closing ``}''.
% Additional authors and addresses can be added with ``\and'',
% just like the second author.
% To save space, use either the email address or home page, not both

\author{Authors Anonymous}


\maketitle
%\thispagestyle{empty}

%%%%%%%%% ABSTRACT
\begin{abstract}
   pass
\end{abstract}

%%%%%%%%% BODY TEXT
\section{Introduction}

\textbf{The importance of our research area} \\

\textbf{Some progress in sketch based image retrieval} \\

\textbf{Difficulty in Zero-shot setup and some possible solutions} \\

\textbf{Our proposed methods and their advantages} \\

\textbf{itemize our contributions in this paper} \\




%\begin{figure}[t]
%\begin{center}
%\fbox{\rule{0pt}{2in} \rule{0.9\linewidth}{0pt}}
   %\includegraphics[width=0.8\linewidth]{egfigure.eps}
%\end{center}
%   \caption{Example of caption.  It is set in Roman so that mathematics
%   (always set in Roman: $B \sin A = A \sin B$) may be included without an
%   ugly clash.}
%\label{fig:long}
%\label{fig:onecol}
%\end{figure}




%\begin{figure*}
%\begin{center}
%\fbox{\rule{0pt}{2in} \rule{.9\linewidth}{0pt}}
%\end{center}
%   \caption{Example of a short caption, which should be centered.}
%\label{fig:short}
%\end{figure*}

\section{Related Work}

\subsection{Sketch-based image retrieval}

\subsection{Zero-Shot Learning}

\subsection{Disentangled Representation}

\section{Methodology}
There will be five parts in this section. Sec. \ref{3.1} defines the our targeted problem and briefly introduce our framework. Sec.\ref{3.2} introduce the encoders in our model. Sec. \ref{3.3} introduce the decoder in our model. Sec. \ref{3.4} introduce the discriminator in our model. Sec. \ref{3.5} introduce the design of loss functions during training procedure.

\subsection{Problem Definition and Overall model}\label{3.1}
In this paper, we focus on solving the problem of hand-free sketch-based image retrieval using disentangled feature representation under zero-shot setup, where only the sketchs and images from seen classes are used during training stage. Our proposed framework is expected to use the sketchs to retrieve the images, the categories of which have never appeared during training.

We first provide a definition of the SBIR in zero-shot setup. Given a dataset $S=\{(x_i^{img}, x_i^{ske}, x_i^{sem}, y_i)|y_i \in \mathcal{Y}\}$, where $x_i^{img}$, $x_i^{ske}$, $x_i^{sem}$ and $y_i$ corresponding to the image, sketch, semantic representation and 

\subsection{Encoder}\label{3.2}

\subsection{Generator}\label{3.3}

\subsection{Discriminator}\label{3.4}

\subsection{Loss Function}\label{3.5}



%\begin{table}
%\begin{center}
%\begin{tabular}{|l|c|}
%\hline
%Method & Frobnability \\
%\hline\hline
%Theirs & Frumpy \\
%Yours & Frobbly \\
%Ours & Makes one's heart Frob\\
%\hline
%\end{tabular}
%\end{center}
%\caption{Results.   Ours is better.}
%\end{table}

\section{Experiment}

\subsection{Experiment Setup}

\subsubsection{Dataset}

\subsubsection{Implementation Details}

\subsection{Comparison}

\subsection{Ablation Study}

\subsection{Case study}

\section{Conclusion}

\section{To Discuss}
\begin{itemize}
	\item Whether to generator the whole image/sketch.
	\item If the poses between the image and the sketch are different, can the model learn the sketch information between image and sketch.
	\item Where to add the semantics information to further supervise the model's training.
\end{itemize}


{\small
\bibliographystyle{ieee_fullname}
\bibliography{egbib}
}

\end{document}
